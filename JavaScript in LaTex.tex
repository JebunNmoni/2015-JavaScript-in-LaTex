\documentclass[a4paper,12pt]{article}
\usepackage{hyperref}
\usepackage[left=1.5cm,right=1.5cm,top=1.5cm,bottom=1.5cm,ignoreheadfoot]{geometry}
\usepackage{array}
\usepackage[svgnames,table]{xcolor}
\usepackage[pdftex]{insdljs}

\newcommand*{\arraycolor}[1]{\protect\leavevmode\color{#1}}
\newcolumntype{A}{>{\columncolor{blue!50!white}}c}
\newcolumntype{B}{>{\columncolor{LightGoldenrod}}c}
\newcolumntype{C}{>{\columncolor{Green!50}}c}
\OpenAction{\JS{%
var d1=this.getField("dhar1");
var d2=this.getField("dhar2");
var d3=this.getField("dhar3");
this.getField("AmiPai").value = d1.value + d2.value + d3.value;
var f1=this.getField("ferot1");
var f2=this.getField("ferot2");
var f3=this.getField("ferot3");
this.getField("FerotDisos").value = f1.value + f2.value + f3.value;
}}   
\begin{document}    
\section*{Table}
The following table takes the input and shows the output in a pdf file.
\renewcommand{\MakeTextField}[2]{{\vbox to #2{\vfill\hbox to #1{\hrulefill}}}}
\begin{Form}
\begin{center}
\sffamily
\arrayrulecolor{white}
\arrayrulewidth=1pt
\renewcommand{\arraystretch}{1.5}
%\rowcolors[\hline]{3}{.!50!White}{}
\vspace{0.5in}
\begin{tabular}{A|B|B|B|B}
  \multicolumn{4}{C}{\bfseries Calculate Savings}\\
  \rowcolor{.!50!Black}
  \arraycolor{White}\bfseries Month &
  \arraycolor{White}\bfseries Earnings &
  \arraycolor{White}\bfseries Expenditures&
  \arraycolor{White}\bfseries Saved\\
  January &   \TextField[name=JanEr, value=500]{} & 
  				  \TextField[name=JanEx,, value=350]{}& 
  				  \TextField[name=JanS,, value=]{}\\
  February &   \TextField[name=FebEr, value=550]{} & 
  				  \TextField[name=FebEx,, value=700]{}& 
  				  \TextField[name=FebS,, value=]{}\\
  March &   \TextField[name=MarEr, value=700]{} & 
  				  \TextField[name=MarEx,, value=690]{}& 
  				  \TextField[name=MarS,, value=]{}\\
  \rowcolor{.!0!Black}
  \arraycolor{White}\bfseries Total &
  \arraycolor{White}\bfseries \TextField[name=TotalEr]{} &
  \arraycolor{White}\bfseries \TextField[name=TotalEx]{}&
  \arraycolor{White}\bfseries \TextField[name=TotalS]{}\\
\end{tabular}
\end{center}

\vspace{0.2in}

\begin{center}
\PushButton[onclick={ 
					var ErJ=this.getField("JanEr");
					var ExJ=this.getField("JanEx");
					var ErF=this.getField("FebEr");
					var ExF=this.getField("FebEx");
					var ErM=this.getField("MarEr");
					var ExM=this.getField("MarEx");
					this.getField("TotalEr").value = ErJ.value + ErF.value + ErM.value;
					this.getField("TotalEx").value = ExJ.value + ExF.value + ExM.value;
					this.getField("JanS").value = ErJ.value - ExJ.value;
					this.getField("FebS").value = ErF.value - ExF.value;
					this.getField("MarS").value = ErM.value - ExM.value;
					this.getField("TotalS").value = (ErJ.value - ExJ.value) + (ErF.value - ExF.value) + (ErM.value - ExM.value);
					}]{Click Me to Calculate the Total}					
\end{center}

\end{Form}
\end{document}